\documentclass[fleqn,twoside]{article}
\usepackage{espcrc1}

% change this to the following line for use with LaTeX2.09
% \documentstyle[twoside,fleqn,espcrc2]{article}

% if you want to include PostScript figures
\usepackage[ddmmyyyy]{datetime}
\usepackage{graphicx}
\textheight = 20cm
\begin{document}

% declarations for front matter
\title{
  AMS note: \today \\
  \hspace{1cm} \\
  Flux calculation
}

\author{
  B. Coste\address[Trento]{INFN, Istituto Nazionale di Fisica Nucleare, Trento, Italia}
  %Auth1\addressmark[Pisa],
  %Auth2\addressmark[Beijing]
}

% typeset front matter (including abstract)
\maketitle

\begin{figure}[htb]
  \centering
  \includegraphics{plots/japbcvpavbzau9dbuaxf.jpg}
\end{figure}


\section{Introduction}
Let's take a simple example where cosmic rays are only composed of 2 mono-energetic species A and B.
\begin{itemize}
\item We wan't to compute the flux of species A.
\item During one second, the detector has been crossed by $N_{A}$ (resp. $N_{B}$) element of species A (resp. B).
\item B is so abundant that the detector cannot trig everytime one B element cross it. To cope with this problem the detector has 2 triggers:
  an unbiased trigger (called U-trigger in the following) that sees everything (it sees the $N_{A}+N_{B}$ particles passing trough) but triggers only on one over 100 particles it sees. But if we had only this one we would miss a lot of A so we have an other one (let's call it S for science trigger) that triggers quite often when a A go through it. Let's also remind that it is possible that both triggers activate for the same event.
\end{itemize}

From here we can define a few numbers:
\begin{itemize}
\item $N_{A}^{U}$, the number of A trigged by U but not by S.
\item $N_{B}^{U}$, the number of B trigged by U but not by S.
\item $N_{A}^{S}$, the number of A trigged by S but not by U.
\item $N_{B}^{S}$, the number of B trigged by S but not by U.
\item $N_{A}^{U+S}$, the number of A trigged by both triggers.
\item $N_{B}^{U+S}$, the number of B trigged by both triggers.
\end{itemize}

We can also define so efficiencies:
\begin{itemize}
\item $\epsilon_{A}^{U} = \epsilon_{B}^{U} = 1 \%$, the effiency of trigger U for spicies A and B
\item $\epsilon_{A}^{S}$ (resp. $\epsilon_{B}^{S}$) the efficiencies of trigger S for A (resp. B).
\end{itemize}

Now we can lie a couple of relations:

\begin{equation}
  \epsilon_{A}^{U} = \frac{N_{A}^{U} + N_{A}^{U+S}}{N_{A}} = 1 \%
\end{equation}


\begin{equation}
  \epsilon_{A}^{S} = \frac{N_{A}^{S} + N_{A}^{U+S}}{N_{A}}
\end{equation}

And the same stands for B as well.




\section{Unbiased cut}
The raw data are unusable without cleaning. Let's apply a first cut $C_{U}$. We'll call it the unbiased cut and make the hypothesis that its effiency is the same for every species. We will now compute the efficiency $\epsilon_{CU}$ of such a cut.

\subsection{Efficiency of the unbiased cut}
Let's define two new numbers: the numbers of A (resp. B) in our data (ie. really present in the ROOT files) will be $N_{A}^{raw}$ (resp. $N_{B}^{raw}$). It is simply:
\begin{equation}
  \label{rawSample}
  N_{A}^{raw} = N_{A}^{U} + N_{A}^{S} + N_{A}^{U+S}
\end{equation}

The unbiased cut has reduced our A sample to:
$N_{A}^{after C_{U}} = \epsilon_{CU} N_{A}^{raw}$

\subsubsection{What NOT to say: $\epsilon_{CU} = \frac{N_{A}^{after C_{U}} + N_{B}^{after C_{U}}}{N_{A}^{raw} + N_{B}^{raw}}$}

We need to look at probabilities. Let's define $P(Cut)$ the probability that a particle passes the cut.
We have already divided the A sample into 3 sub-samples in equation \ref{rawSample} so let's have a look at how many particles of every sub-sample pass the cut.

$N_{A}^{U after C_{U}} = N_{A}^{U} P(cut)$ is valid but what is not is the same for $N_{A}^{S}$ and $N_{A}^{U+S}$ sub-samples.
For them the equation is:
\begin{equation}
  N_{A}^{S after C_{U}} = N_{A}^{S} P(cut|S)
\end{equation}

where $P(cut|S)$ is the probability that a particle passes the unbiased cut \textbf{given that} the particle triggers the S trigger. This is not uncorrelated since it means that the particle exhibits a particular behaviour. For example it can have a better defined TrTrack and that raises the probability of passing the cut with respect to a random particle.

\subsubsection{The proper answer: $\epsilon_{CU} = \frac{N_{A}^{U after C_{U}} + N_{B}^{after C_{U}}}{N_{A}^{raw} + N_{B}^{raw}}$}

\subsection{Discussion: is unbiased cut really unbiased ?}
At some point we should discuss that.


\begin{thebibliography}{99}

\end{thebibliography}

\end{document}

